\documentclass[journal]{vgtc}                % final (journal style)
%\documentclass[review,journal]{vgtc}         % review (journal style)
%\documentclass[widereview]{vgtc}             % wide-spaced review
%\documentclass[preprint,journal]{vgtc}       % preprint (journal style)
%\documentclass[electronic,journal]{vgtc}     % electronic version, journal

%% Uncomment one of the lines above depending on where your paper is
%% in the conference process. ``review'' and ``widereview'' are for review
%% submission, ``preprint'' is for pre-publication, and the final version
%% doesn't use a specific qualifier. Further, ``electronic'' includes
%% hyperreferences for more convenient online viewing.

%% Please use one of the ``review'' options in combination with the
%% assigned online id (see below) ONLY if your paper uses a double blind
%% review process. Some conferences, like IEEE Vis and InfoVis, have NOT
%% in the past.

%% Please note that the use of figures other than the optional teaser is not permitted on the first page
%% of the journal version.  Figures should begin on the second page and be
%% in CMYK or Grey scale format, otherwise, colour shifting may occur
%% during the printing process.  Papers submitted with figures other than the optional teaser on the
%% first page will be refused.

%% These three lines bring in essential packages: ``mathptmx'' for Type 1
%% typefaces, ``graphicx'' for inclusion of EPS figures. and ``times''
%% for proper handling of the times font family.

\usepackage{mathptmx}
\usepackage{graphicx}
\usepackage{times}

%% We encourage the use of mathptmx for consistent usage of times font
%% throughout the proceedings. However, if you encounter conflicts
%% with other math-related packages, you may want to disable it.

%% This turns references into clickable hyperlinks.
\usepackage[bookmarks,backref=true,linkcolor=black]{hyperref} %,colorlinks
\hypersetup{
  pdfauthor = {},
  pdftitle = {},
  pdfsubject = {},
  pdfkeywords = {},
  colorlinks=true,
  linkcolor= black,
  citecolor= black,
  pageanchor=true,
  urlcolor = black,
  plainpages = false,
  linktocpage
}

%% If you are submitting a paper to a conference for review with a double
%% blind reviewing process, please replace the value ``0'' below with your
%% OnlineID. Otherwise, you may safely leave it at ``0''.
\onlineid{0}

%% declare the category of your paper, only shown in review mode
\vgtccategory{Research}

%% allow for this line if you want the electronic option to work properly
\vgtcinsertpkg

%% In preprint mode you may define your own headline.
%\preprinttext{To appear in IEEE Transactions on Visualization and Computer Graphics.}

%% Paper title.

\title{TRIAD: Towards Real-time, Interactive Analysis of Data}

%% This is how authors are specified in the journal style

%% indicate IEEE Member or Student Member in form indicated below
\author{Remco Chang \and Danyel Fisher \and Jeffrey Heer \and Carlos Scheidegger}
\authorfooter{
%% insert punctuation at end of each item
\item
 Remco Chang is with Tufts University. email: remco@cs.tufts.edu
\item
 Danyel Fisher is with Microsoft Research. email: danyelf@microsoft.com
\item
 Jeffrey Heer is with the University of Washington. email: jheer@uw.edu
\item
 Carlos Scheidegger is with the University of Arizona. email: cscheid@cs.arizona.edu
}

%other entries to be set up for journal
\shortauthortitle{Workshop proposal: TRIAD}
%\shortauthortitle{Firstauthor \MakeLowercase{\textit{et al.}}: Paper Title}

%% Abstract section.
\abstract{}

%% Keywords that describe your work. Will show as 'Index Terms' in journal
%% please capitalize first letter and insert punctuation after last keyword
\keywords{}

%% ACM Computing Classification System (CCS). 
%% See <http://www.acm.org/class/1998/> for details.
%% The ``\CCScat'' command takes four arguments.

\CCScatlist{}

%% Uncomment below to disable the manuscript note
%\renewcommand{\manuscriptnotetxt}{}

%% Copyright space is enabled by default as required by guidelines.
%% It is disabled by the 'review' option or via the following command:
% \nocopyrightspace

%%%%%%%%%%%%%%%%%%%%%%%%%%%%%%%%%%%%%%%%%%%%%%%%%%%%%%%%%%%%%%%%
%%%%%%%%%%%%%%%%%%%%%% START OF THE PAPER %%%%%%%%%%%%%%%%%%%%%%
%%%%%%%%%%%%%%%%%%%%%%%%%%%%%%%%%%%%%%%%%%%%%%%%%%%%%%%%%%%%%%%%%

\begin{document}
\maketitle
%% The ``\maketitle'' command must be the first command after the
%% ``\begin{document}'' command. It prepares and prints the title block.
\section{Goals}
%% the only exception to this rule is the \firstsection command
The goal of this workshop is to foster innovative work in the backends
that will support the next generation of interactive data analysis
tools. We envision such backends to require novel insights in database
technology, algorithm design, and information visualization techniques.

In the current design of database and analytics systems, interactive
data analysis concerns are usually an afterthought. Although recent
architectural changes like columnar databases have enabled faster
visualization systems, these sorts of innovations have been designed
less for visualization than for accelerating queries in general. In
this workshop, we will explore the consequences of pushing the
concerns of interactive data analysis deep into the ``computational
fabric'' of our systems: the back-end infrastructure of data storage,
organization, and retrieval. If we design with the knowledge that
every analysis is eventually read by a human, bound by perceptual
limitations and latency constraints, what are the consequences for
database design, or data analysis and visualization algorithms? Are
there fundamental tradeoffs? What can we give up?

There have been previous workshops in VIS that have investigated
issues of large data visualization, such as LDAV (and LDAV will be a
co-located symposium this year at VIS). Our focus is markedly
different from that of LDAV. LDAV focuses on simulation and imaging
data, while we believe current techniques in information visualization
and visual analytics are close to hitting a ceiling in terms of
performance. We propose this workshop as a venue to discuss these
limits, and fundamental new ways to attack them.

\section{Technical Scope}
The scope of this workshop will span, among possibly other fields,
information visualization, visual analytics, algorithm design,
database architecture, and statistics. Although this is clearly a
broad scope, the workshop will focus on the backends for interactive
tools. We have made significant progress in our understanding of
perceptual and human-interaction issues behind interactive
visualization design. However, much of this understanding has not, in
general, been translated into new designs for our backend
infrastructure. We think now is the right time to have a discussion of
backend design for interactive data analysis. Following the principle
of ``No One Size Fits All'' in designing databases [7], recent
publications by the organizers have shown that by taking advantage of
needs and limitations of interactive visualization systems, databases
can be optimized by making tradeoffs in novel and interesting
ways. For example, imMens [5], profiler [2], and nanocubes [4] have
shown that careful algorithm design can enable extremely fast
interactive visualizations of massive data. Scalr [3] has shown that
traditional query planners can fruitfully take advantage of perceptual
and resolution limits. ForeCache [6] utilizes a predictive prefetching
strategy by learning a user’s interaction patterns and prefetching
data from server to support real-time visual exploration. Lastly,
SampleAction [1] takes advantage of streaming databases and
incorporates HCI principles that users need immediate feedback, but
are willing to wait for high-fidelity results.

In addition to the work by the organizers, researchers in the database
community have also made significant stride in developing databases
specifically for interactive visualization, analysis, and exploration
of data. For example, systems like Tableau utilizes columnar databases
which provide speedups for typical visual analytics tasks that require
operations across columns (instead of the traditional rows). BlinkDB
[8] uses a sampling strategy to build a smaller, but representative
database from a larger one. Map-D [9] takes advantage of modern
graphics card capabilities and resulted in a specialized database that
runs entirely within (an array of) graphics cards. Finally, SciDB [10]
breaks away from the traditional relational data structure and is
designed to store and query multidimensional scientific data that are
stored in an array-based structure.

Given the increasing number of publications and backend systems that
can be tailored or applied to interactive visual exploration of large
data, we believe that it is very important for the InfoVis and the
VAST communities to begin having discussions and to develop a research
agenda that focuses on the backend technologies. In this proposed
workshop, we seek to invite members from the database community
(including authors of the aforementioned papers) to meet with the VIS
community and begin such dialogs. As many of these database
researchers are current collaborators of the organizers (see Section
XXX below for a list of intended invitees), we believe that the
organizing team is well suited to influence research not only in
information visualization, but in these other related fields as well.

\section{Format, Length, and Planned Activities}
We propose a half-day workshop that focuses on discussion between the
VIS and database researchers. The workshop will solicit submissions of
position papers as well as short research papers from both the VIS and
database communities. These papers will be reviewed by the organizing
team. The authors of the selected top papers will be asked to each
give a short presentation of their work (or position).

The goal of the presentation is to allow the attendees to get an
overview of the different perspectives and approaches of data storage,
organization, and retrieval techniques that can be tailored
specifically to support interactive data visualization and
exploration. We anticipate that these presentations will take up
approximately the first half of the workshop.

Following a short break, the second half of the workshop will be
oriented around open discussions and possibly a question-and-answer
session. As we anticipate that members of the two different
communities will not have a common language or viewpoint, we believe
that the open-ended format will facilitate discussions and general
knowledge sharing.

It is relevant to note that the organizing team is also currently
seeking a followup ``closed-door'' (invitation only) meeting that will
take place in the afternoon outside of the main VIS venue. The purpose
of this closed-door meeting is to allow for a more intimate discussion
between enthusiastic researchers who are actively seeking
collaborations across the aisle. Although it is not the goal of the
organizing team to be exclusive, it is our belief that a more focused
discussion will have a higher likelihood towards developing a cohesive
research agenda. Note that this is not an extraordinary proposal: such
a format has been successfully employed by other workshops in previous
years at VIS, most notably by the BELIV workshop~\cite{beliv}.

\section{List of Participants and Invitees}

As noted earlier, it is our hope that this workshop will include both
attendees from the VIS and database communities. However, since the
workshop will be located at VIS, we will specifically reach out to
members of the database community and invite them to attend the
workshop. We will seek a mix of researchers in different stages of
their careers, together with researchers from different companies in
the industry. If possible, we will also seek to identify a more senior
person from this invited list as the keynote speaker for the workshop.

Senior researchers:

\begin{itemize}
\item Sam Madden (MIT)
\item Joe Hellerstein (Berkeley, Trifacta)
\item Mike Stonebraker (MIT)
\item Ugur Cetintemel (Brown)
\item Stan Zdonik (Brown)
\end{itemize}

Early or mid-career researchers:
\begin{itemize}
\item Eugene Wu (Columbia)
\item Aaron Elmore (U. Chicago)
\item Joey Gonzalez (Berkeley AMPLab, CMU, GraphLab/Dato)
\item Bill Howe (UW)
\item Arnab Nandi (OSU)
\item Cecilia Aragon (UW; astronomy/scivis background)
\item Jenny Duggan (Northwestern)
\item Mike Cafarella (Michigan)
\end{itemize}

Junior Researchers: (Students, etc)
\begin{itemize}
\item Manasi Vartak (MIT, SeeDB)
\item Todd Mostak (MapD)
\item Leilani Battle (MIT, Tile Prefetching)
\item Dominik Moritz (UW)
\item Holger Pirk (MIT postdoc, databases)
\end{itemize}

Relevant companies we will reach out to:
\begin{itemize}
\item Tableau
\item Trifacta
\item Paradigm4
\item Graphistry
\end{itemize}

\section{Requested Facilities}

We should ask for a setup split halfway with seats for lectures and space for posters in the back. TBD.

\section{Intended Result and Impact of Workshop Results}

TBD.

\section{Related publications by Organizers}
\cite{Battle:2013:DRO, Lins:2013:NFR, Liu:2013:IRT, Fisher:2012:TMI, Kandel:2012:PIS}
[6] Battle, Chang, Stonebraker. Dynamic Prefetching of Data Tiles for Interactive Visualization. VLDB, 2015 (Conditionally Accepted).

\section{References}

\cite{Agarwal:2013:BQW}

[7] Stonebraker and Cetintemel. "One Size Fits All": An Idea Whose Time Has Come and Gone. ICDE '05
[9] Mostak. An overview of MapD. http://map-d.com/docs/mapdwhitepaper. 2014
[10] Cudre-Mauroux et al. SciDB: A Science-Oriented DBMS. VLDB 2009.

\section{Organizer Bios}

\paragraph*{Remco Chang} is an Assistant Professor in the Computer Science Department at Tufts University. He received his BS from Johns Hopkins University in 1997 in Computer Science and Economics, MSc from Brown University in 2000, and PhD in computer science from UNC Charlotte in 2009. Prior to his PhD, he worked for Boeing developing real-time flight tracking and visualization software, followed by a position at UNC Charlotte as a research scientist. His current research interests include visual analytics, information visualization, and HCI. His research has been funded by NSF, DHS, MIT Lincoln Lab, and Draper. He received the NSF CAREER Award in 2015.

\paragraph*{Danyel Fisher} is a Senior Researcher in information visualization and human-computer interaction at Microsoft Research's VIBE group. His research focuses on ways to help users interact with data more easily. His recent work has looked at ways to make big data analytics faster and more interactive. Danyel received his MS from UC Berkeley, and his PhD from UC Irvine. He tweets at @FisherDanyel.

\paragraph*{Jeffrey Heer} is an Associate Professor of Computer Science \& Engineering at the University of Washington, where he directs the Interactive Data Lab and conducts research on data visualization, human-computer interaction and social computing. The visualization tools developed by his lab (D3.js, Vega, Protovis, Prefuse) are used by researchers, companies and thousands of data enthusiasts around the world. His group's research papers have received awards at the premier venues in Human-Computer Interaction (ACM CHI, UIST, CSCW) and Information Visualization (IEEE InfoVis, VAST, EuroVis). Other awards include MIT Technology Review's TR35 (2009), a Sloan Foundation Research Fellowship (2012), and a Moore Foundation Data-Driven Discovery Investigator award (2014). Jeff holds BS, MS and PhD degrees in Computer Science from UC Berkeley, whom he then betrayed by joining the Stanford faculty (2009-2013). Jeff is also a co-founder of Trifacta, a provider of interactive tools for scalable data transformation.

\paragraph*{Carlos Scheidegger} is an Assistant Professor in the Department of Computer Science at the University of Arizona since 2014. Prior to joining the Arizona faculty, Carlos worked at AT\&T Research from 2009 to 2014, where he helped develop award-winning, open-source techniques for massive data sources (http://nanocubes.net). His current research interests are in the intersection of large-scale data analysis, information visualization, data management, and software infrastructure for scientific collaboration. His research has received multiple awards at top venues in data visualization (IEEE InfoVis, IEEE SciVis, EuroVis).

\bibliographystyle{abbrv}
\bibliography{paper}
\end{document}

